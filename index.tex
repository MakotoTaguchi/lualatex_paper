%%%%%%初期設定 start%%%%%%%%%%%%%%%%%%%%%%%%%%%%%%%%%%%%%%%%%%%%%

\documentclass[12pt]{ltjsreport}%LuaLaTeX対応
\renewcommand{\bibname}{参考文献} 
\usepackage{listings,jvlisting,graphicx, ascmac, url, array, ./style/eclbkbox,fancyheadings}
\usepackage[luatex,pdfencoding=auto]{hyperref}
\hypersetup{unicode,bookmarksnumbered=true,hidelinks,final}

\lstset{
  basicstyle={\ttfamily},
  identifierstyle={\small},
  commentstyle={\smallitshape},
  keywordstyle={\small\bfseries},
  ndkeywordstyle={\small},
  stringstyle={\small\ttfamily},
  frame={tb},
  breaklines=true,
  columns=[l]{fullflexible},
  numbers=left,
  % xrightmargin=0zw,
  % xleftmargin=3zw,
  numberstyle={\scriptsize},
  stepnumber=1,
  % numbersep=1zw,
  lineskip=-0.5ex
}

\setlength{\textheight}{25cm}		%1ページ当りの行数を指定する
\setlength{\textwidth}{38\zw}		%1行あたりの文字数の設定
\setlength{\evensidemargin}{10mm}   %偶数ページの余白
\setlength{\oddsidemargin}{10mm}    %奇数ページの余白
\setlength{\topmargin}{-3mm}        %上の余白
\setlength{\headheight}{0mm}        %ヘッダ領域の高さ


\setlength{\headsep}{0mm}           %ヘッダ領域と本文領域との間隔
\setlength{\columnsep}{12mm}        %段組にした場合の段同士の間隔
\setlength{\footskip}{10mm}         %フッタ領域と本文との間隔

%%%%%%%タイトル内容 start%%%%%%%%%%%%%%%%%%%%%%%%%%%%%%%%%%%%%%%%%%

% \title{FaceAPIを用いた表情分析を活用した\\労務管理支援の提案}					%卒業論文タイトル
% \author{1932028 片平 純太郎\\1932080 田口 真人\\\normalsize 指導教員:中村 直人 教授}	%名前(苗字と名前は全角1字空け)
% \date{令和4年度}                                %日付設定(デフォルトは現在の年月日)

%%%%%%%タイトル内容 end%%%%%%%%%%%%%%%%%%%%%%%%%%%%%%%%%%%%%%%%%%%

%%%%%%%初期設定 end%%%%%%%%%%%%%%%%%%%%%%%%%%%%%%%%%%%%%%%%%%%%%

%%%%%%%コンテンツ start%%%%%%%%%%%%%%%%%%%%%%%%%%%%%%%%%%%%%%%%%%%%
\begin{document}

\begin{titlepage}
  \begin{center}
  \vspace*{50truept}
  {\Huge 千葉工業大学}\\
  \vspace{20truept}
  % {\Huge 修士学位論文}\\
  {\Huge 卒業論文}\\
  \vspace{170truept}
  {\Huge FaceAPIを用いた表情分析を活用した労務管理支援の提案} \\
  \vspace{30truept}
  {\Large 2023年3月}\\
  \vspace{150truept}
  % {\Large 所属先行:情報科学専攻}\\
  {\Large 所属学科:情報ネットワーク学科}\\
  \vspace{10truept}
  {\Large 学生番号・氏名:}\\
  \vspace{10truept}
  {\Large 1932028番 片平 純太郎}\\
  {\Large 1932080番 田口 真人}\\
  \vspace{25truept}
  {\Large 指導教員:中村 直人 教授}
  \end{center}
  \end{titlepage}

\pagenumbering{roman}                         %タイトルをドキュメントへ貼り付け
\tableofcontents                 %目次を作成
\listoffigures				 %図の目次作成
\listoftables				 %表の目次作成

\baselineskip 20pt               %行間設定


\clearpage
\pagenumbering{arabic}

%%%%%%%本文 start%%%%%%%%%%%%%%%%%%%%%%%%%%%%%%%%%%%%%%%%%%%

%別ファイルを読み込み
%\input{ディレクトリ名/ファイル名}
%http://www.latex-cmd.com/
\chapter{序論}
\label {chp:tex_basic}

\section{研究の背景と目的}
\label{sec:tex_basic_section}
昨今,コロナの影響により,人と直接会う機会が減り,マスクの着用を強いられるようになった.
その結果,笑顔になる機会が減るため,笑顔の減少に繋がる. \\
 しかし,総合人材サービスのパーソルホールディングス株式会社が行った調査[1]では,
仕事をする上で笑顔になると「楽しい」という気持ちが高まった人は約6割で,
ポジティブな感情状態で仕事に取り組んでいた人ほど,笑顔になっており,
職場において笑顔が高まれば,自発的に取り組む傾向があるという結果が出た.

\begin{figure}[!h]
    \begin{center}
        \includegraphics[scale=0.5, clip]{./img/work.png}
        \caption{笑顔計測後の主な感情変化}
        \label{fig:図の名前}
    \end{center}
\end{figure}

また,厚生労働省の調査[2]によると,下のグラフから分かるように,
平成19年度から令和元年にかけて約15\%も増加している.

\clearpage

\begin{figure}[!h]
    \begin{center}
        \includegraphics[scale=0.6, clip]{./img/graph.png}
        \caption{厚生労働省によるパワハラ調査}
        \label{fig:図の名前}
    \end{center}
\end{figure}

そこで我々は,以上の課題である「笑顔の減少」と「パワハラの増加」の解決を目的に
FaceAPIを用いた表情分析を活用した労務管理支援システムを提案する.

\section{論文の構成}
\label{sec:tex_basic_newline}
本論文は以下のような構成になっている.
\\
第 1 章では研究背景を述べる.
\\
第 2 章では本研究で開発する労務管理支援システムの概要を述べる.
\\
第 3 章では本研究で開発する労務管理支援システムの構成を述べる.
\\
第 4 章では本研究で開発する労務管理支援システムの実装結果を述べる.
\\
第 5 章では本研究のまとめを述べる.

\chapter{機能設計}
\label{chp:chart}

\section{FaceAPIを用いた労務管理支援システム}
\label{sec:chart_figure}
今回作成するシステムは,会社で使うことを想定しており,労務管理者と社員で使える機能を分ける.
次の章からそれぞれの機能について述べる.

\section{管理者用の機能}
\label{sec:chart_admin}
\subsection{社員管理}
社員管理は社員の名前,メールアドレス,社員番号などを表で管理する.さらにその表から名前などを編集
できる機能を実装する.

\subsection{掲示板}
労務管理者が社員に何かお知らせする際に利用できる.管理者しか発言できないようになっている.



\chapter{システムの実装方法}
\label{chp:reference}

\section{システム構成}
\label{sec:reference_ftnote}
本システムは以下のような構成で実装を行う.システム構成図を図3.1に示す.\\
労務管理支援システムをReactでWebアプリとして作成した.さらにデータベースにCloud Firestore,
ユーザー認証にFirebase Authentication,表情分析にface-api.jsを使用した。\\
 次のセクションでは,それぞれのサービスについて説明する.

\begin{figure}[!h]
	\begin{center}
			\includegraphics[scale=1.2, clip]{./img/compose.png}
			\caption{システム構成図}
			\label{fig:図の名前}
	\end{center}
\end{figure}


\section{React}
\label{sec:reference_quote}
Reactは,UI(ユーザインタフェース)部分の構築に特化したJavaScriptのライブラリで,
React.jsとも呼ばれる.
SNSで有名なMeta社(旧Facebook社)が自社サービスの機能拡張に伴うコードの複雑化によって
維持管理がしにくくなることを防ぐために開発した.
コーディングコストが少なく,開発規模が大きくなっても管理しやすいといった特長もあり,
現在では開発元であるFacebook社のサービスであるFacebookやInstagramはもちろんのこと,
Yahoo!やAirbnb,Reddit,Netflix,Slack,Uberといった世界的なWebサイトや
Webアプリで利用されるなど,世界中の多くの企業で採用されており,日本でも注目を集める
など,今最も勢いのあるライブラリである.
今回のシステム開発にReactを採用した理由は三つある.

	\begin{enumerate}
		\item パフォーマンスが良い \\
		Reactには,仮想DOM(Virtual Document Object Model)というレンダリング機構が
		備わっている.仮装DOMとは、実際のDOMではなく, React内部に持っている 
		DOMの情報である. Reactを使うと,この仮想DOMと実際のHTML上のDOMを
		比較したときに出てくる違いだけが,毎回HTML上に再適用される.
		そのため画面全体がReactで構成されていたとしても,必要な部分しか更新されず
		非常に高速に動作するため,パフォーマンスが良い.\\

		\item UIコンポーネントのライブラリが多い \\
		Reactは,世界中で使われているため,Reactのライブラリを使ってUIをコンポート化
		するようになってきている.あらかじめButtonやFormなどのUIパーツを
		Reactコンポーネントとして扱えるようにして,セット化したものが多くある.
		これらを使えば,今風の洗練された画面を作ることができる.\\

		\item JavaScriptの知識があれば使える \\
		基本的にReactはJavaScriptで書かれているため,
		JavaScriptの知識があればアプリを開発することができる.
		たとえJavaScriptの開発経験がなくても基本構文を理解していれば開発に
		取り掛かれる.今回,我々はJavaScriptの学習を既に行なっていたので,
		Reactを選んだ.
	\end{enumerate}
	
\section{face-api.jsについて}
\label{sec:reference_bib}
face-api.jsはブラウザ, NodeJSで顔を検出するための,
Tensorflow.jsを活用したJavaScript APIである.
ここでは,face-api.jsがどのような機能を提供しているかを紹介する.
\begin{itemize}
	\item 顔検出 \\
	写真から顔を探して検出する. \\

	\item 顔のランドマーク検出 \\
	検出した顔の目や鼻の位置など,顔の特徴を抽出する上で重要なキーポイントを検出する. \\

	\item 表情認識 \\
	検出した顔の表情を認識する.表情は,「怒り」,「嬉しさ」,「中立」,「恐怖」,「うんざり」,
	「驚き」,「悲しさ」の7種類あり,数値で表される. \\

	\item 年齢推定 \\
	検出した顔の年齢を推定する. \\

	\item 性別認識 \\
	検出した顔の性別を推定する. \\

	今回は,顔検出と表情認識の機能を使い,表情分析した値をグラフ化することを目的とする.
\end{itemize}
	
\section{Material-UIについて}
\label{sec:reference_chapter}



\chapter{動作検証}
\label{chp:tex_pro}
この章では動作検証の結果をスクリーンショットとソースコードを元に紹介する.
今回はGoogle Chrome上で動作させるため,他のWebブラウザ上ではUIが異なっている可能性がある.

\section{社員が写真を提出する}
\label{sec:tex_pro_cmd}
社員ページに顔写真提出するためのフォームを作成した.写真提出ボタンから提出する写真を選択し,
提出する.その実行しているスクリーンショットを図4.1,図4.2,図4.3,図4.4に示す.
face-api.jsを用いて,表情を分析し,それらの値と笑顔の値などに応じて加算される笑みポイントを
useStateで保存する(ソースコード4.1).
\\

\renewcommand{\lstlistingname}{ソースコード}

\begin{lstlisting}[caption=表情分析]
  const handleImage = async () => {
    const detections = await faceapi
      .detectAllFaces(imgRef.current, new faceapi.TinyFaceDetectorOptions())
      .withFaceExpressions();

    if (detections[0].expressions.happy >= 0.7) {
      setObject({
        expressions: {
          angry: detections[0].expressions.angry,
          disgusted: detections[0].expressions.disgusted,
          fearful: detections[0].expressions.fearful,
          happy: detections[0].expressions.happy,
          neutral: detections[0].expressions.neutral,
          sad: detections[0].expressions.sad,
          surprised: detections[0].expressions.surprised,
        },
        point: 3,
      });
    } else if (
      detections[0].expressions.happy >= 0.5 ||
      detections[0].expressions.surprised >= 0.7 ||
      detections[0].expressions.neutral >= 0.7
    ) {
      setObject({
        expressions: {
          angry: detections[0].expressions.angry,
          disgusted: detections[0].expressions.disgusted,
          fearful: detections[0].expressions.fearful,
          happy: detections[0].expressions.happy,
          neutral: detections[0].expressions.neutral,
          sad: detections[0].expressions.sad,
          surprised: detections[0].expressions.surprised,
        },
        point: 2,
      });
    } else if (
      detections[0].expressions.happy >= 0.3 ||
      detections[0].expressions.surprised >= 0.5 ||
      detections[0].expressions.neutral >= 0.5
    ) {
      setObject({
        expressions: {
          angry: detections[0].expressions.angry,
          disgusted: detections[0].expressions.disgusted,
          fearful: detections[0].expressions.fearful,
          happy: detections[0].expressions.happy,
          neutral: detections[0].expressions.neutral,
          sad: detections[0].expressions.sad,
          surprised: detections[0].expressions.surprised,
        },
        point: 1,
      });
    } else {
      setObject({
        expressions: {
          angry: detections[0].expressions.angry,
          disgusted: detections[0].expressions.disgusted,
          fearful: detections[0].expressions.fearful,
          happy: detections[0].expressions.happy,
          neutral: detections[0].expressions.neutral,
          sad: detections[0].expressions.sad,
          surprised: detections[0].expressions.surprised,
        },
        point: 0,
      });
    }
  };

  useEffect(() => {
    const loadModels = () => {
      Promise.all([
        faceapi.nets.tinyFaceDetector.loadFromUri("/models"),
        faceapi.nets.faceLandmark68Net.loadFromUri("/models"),
        faceapi.nets.faceExpressionNet.loadFromUri("/models"),
      ])
        .then(handleImage)
        .catch((e) => console.log(e));
    };

    imgRef.current && loadModels();
  }, []);
\end{lstlisting}

もし顔認識できない写真であれば,もう一度写真を提出させる処理をすることでエラー回避している
(ソースコード4.2).そのスクリーンショットを図4.5に示す.

\begin{lstlisting}[caption=顔認識エラー処理]
  if (object.length !== undefined) {
    return (
      <div>
        <p>顔認識できません。</p>
        <p
          className="submit reset"
          onClick={() => {
            setNum(1);
          }}
        >
          やり直す
        </p>
      </div>
    );
  }
\end{lstlisting}

\vspace{4mm}

\begin{figure}[!h]
	\begin{center}
			\includegraphics[scale=0.3, clip]{./img/sample1.png}
			\caption{提出画面}
			\label{fig:図の名前}
	\end{center}
\end{figure}

\begin{figure}[!h]
	\begin{center}
			\includegraphics[scale=0.3, clip]{./img/sample2.png}
			\caption{写真選択画面}
			\label{fig:図の名前}
	\end{center}
\end{figure}

\begin{figure}[!h]
	\begin{center}
			\includegraphics[scale=0.3, clip]{./img/sample3.png}
			\caption{画面}
			\label{fig:図の名前}
	\end{center}
\end{figure}

\begin{figure}[!h]
	\begin{center}
			\includegraphics[scale=0.3, clip]{./img/sample4.png}
			\caption{提出完了画面}
			\label{fig:図の名前}
	\end{center}
\end{figure}

\clearpage

\begin{figure}[!h]
	\begin{center}
			\includegraphics[scale=0.3, clip]{./img/sample5.png}
			\caption{顔認識失敗画面}
			\label{fig:図の名前}
	\end{center}
\end{figure}

\section{管理者の社員管理機能}
\label{chp:tex_admin}
管理者ページでは,管理者が社員を管理しやすくするために,表で管理しており,
社員ID,名前,メールアドレス,笑みポイントを可視化している.
図4.6に表が表示されているスクリーンショットを示す.
その他に,それぞれの社員の名前やメールアドレスなどを編集できる機能,提出された写真を見る機能,
社員とトークできる機能がある.それぞれについて,以下で説明する.

\subsection{編集機能}
編集ページのスクリーンショットを図4.7に示す.
編集機能では,社員ID,笑みポイント,役職,名前,メールアドレスを編集できる.
編集した社員IDが他の社員IDと被ってしまうとエラーが起きてしまうので,ソースコード4.3,
ソースコード4.4の様にしてエラー回避している.変数querySnapshot2でidが被っているユーザーを
検索し,変数qに配列として代入している.idが一つでも被っていたらqの大きさは0では無くなるので,
ソースコード4.3の9行目から始まるif文とソースコード4.4でidが被った際の処理をしている.
その際のスクリーンショットを図4.8に示す.

\clearpage

\begin{lstlisting}[caption=社員IDが被った時の処理]
  const handleSubmit = async (event) => {
    event.preventDefault();
    const data = new FormData(event.currentTarget);
    const id = Number(data.get("id"));
    const querySnapshot2 = await getDocs(
      query(collection(db, "users"), where("id", "==", id)));
    const q = querySnapshot2.docs.map((doc) => doc.id);

    if (q.length === 0) {
      setNum(2);
      // db更新
      const querySnapshot = await getDocs(
        query(collection(db, "users"), where("id", "==", props.count))
      );
      const docId = querySnapshot.docs.map((doc) => doc.id).toString();
      await updateDoc(doc(db, "users", docId), {
        id: Number(data.get("id")),
        name: data.get("name"),
        email: data.get("email"),
        point: Number(data.get("point")),
        role: data.get("role"),
      });
    } else {
      setNum(3);
    }
  };
\end{lstlisting}

\begin{lstlisting}[caption=社員IDが被った時の処理]
  else if (num === 3) {
    return (
      <div>
        <ArrowBackRoundedIcon
          sx={{ fontSize: 60 }}
          className="back"
          onClick={() => {setNum(0);}}
        />
        <p>idが被っています</p>
      </div>
    );
  }
\end{lstlisting}

\begin{figure}[!h]
	\begin{center}
			\includegraphics[scale=0.3, clip]{./img/sample6.png}
			\caption{社員管理画面}
			\label{fig:図の名前}
	\end{center}
\end{figure}

\begin{figure}[!h]
	\begin{center}
			\includegraphics[scale=0.3, clip]{./img/sample7.png}
			\caption{編集画面}
			\label{fig:図の名前}
	\end{center}
\end{figure}

\clearpage

\begin{figure}[!h]
	\begin{center}
			\includegraphics[scale=0.3, clip]{./img/sample8.png}
			\caption{idが被っている際の画面}
			\label{fig:図の名前}
	\end{center}
\end{figure}

\subsection{社員の顔写真閲覧機能}
社員の顔写真を閲覧できるページは図4.6の社員管理画面の詳細ボタンから移動できる.
移動すると過去5日分の写真の笑顔度の傾向に関してのグラフを表示している.
図4.9にそのスクリーンショットを示す.
提出されている写真が一枚も無い,または一枚のみの場合は笑顔度の傾向に関してのグラフを
表示しない様にした(図4.10).
日付のボタンは今日から4日前までの5日分あり,押すとそれぞれの日付で提出された社員の顔写真と
その顔を表情分析したグラフが表示される(図4.11).
もしボタンを押した日付の顔写真が提出されていなかったら図4.12の様になる.

\begin{figure}[!h]
	\begin{center}
			\includegraphics[scale=0.3, clip]{./img/sample9.png}
			\caption{顔写真閲覧画面}
			\label{fig:図の名前}
	\end{center}
\end{figure}

\begin{figure}[!h]
	\begin{center}
			\includegraphics[scale=0.3, clip]{./img/sample9.png}
			\caption{笑顔度傾向グラフ非表示画面}
			\label{fig:図の名前}
	\end{center}
\end{figure}

\clearpage

\begin{figure}[!h]
	\begin{center}
			\includegraphics[scale=0.3, clip]{./img/sample10.png}
			\caption{日付のボタンを押した際の画面}
			\label{fig:図の名前}
	\end{center}
\end{figure}

\begin{figure}[!h]
	\begin{center}
			\includegraphics[scale=0.3, clip]{./img/sample11.png}
			\caption{写真が提出されていなかった際の画面}
			\label{fig:図の名前}
	\end{center}
\end{figure}

\clearpage

\section{認証機能}

認証機能はFirebase Authenticationを利用している.今回はユーザーにログインさせる
方法としてGoogleサインインを実装した.ソースコード4.5はそれを実装したプログラムを示している.
これを利用することによって,アカウント作成といった作業の省略や安全性を提供できる.
ログイン画面のスクリーンショットを図4.13,図4.14に,図4.15にログイン完了画面を示す.
\\

\begin{lstlisting}[caption=Googleログインの実装]
  const SignIn = () => {
    // ここからGoogleログイン
  const signInWithGoogle = () => {
    // サインインポップアップ画面を表示
    signInWithPopup(auth, provider).then((result) => {
      // 初めてログインしたかどうかの判定
      const isNewUser = getAdditionalUserInfo(result)?.isNewUser;
      if (isNewUser) {
        addDB();
      } else {
        updateDB();
      }
    });
  };

  // ドキュメント追加
  const addDB = async () => {
    await addDoc(collection(db, "users"), {
      // googleのアカウント名
      name: auth.currentUser.displayName,
      // Gmailアドレス
      email: auth.currentUser.email,
      // ログイン日時
      login: serverTimestamp(),
      // 笑みポイント
      point: 0,
      // 役職
      role: "employee",
      // ユーザーID
      uid: auth.currentUser.uid,
    });
  };

  // ドキュメントフィールド更新
  const updateDB = async () => {
    // ユーザー情報の検索
    const querySnapshot = await getDocs(
      query(collection(db, "users"), where("uid", "==", auth.currentUser.uid))
    );
    // ドキュメントIDの取得
    const docId = querySnapshot.docs.map((doc) => doc.id).toString();
    await updateDoc(doc(db, "users", docId), {
      login: serverTimestamp(),
    });
  };
\end{lstlisting}

\vspace{17mm}

\begin{figure}[!h]
\begin{center}
  \includegraphics[scale=0.3, clip]{./img/sample13.png}
  \caption{ログイン画面}
  \label{fig:図の名前}
\end{center}
\end{figure}

\begin{figure}[!h]
  \begin{center}
    \includegraphics[scale=0.3, clip]{./img/sample14.png}
    \caption{アカウント選択画面}
    \label{fig:図の名前}
  \end{center}
  \end{figure}

  \begin{figure}[!h]
    \begin{center}
      \includegraphics[scale=0.3, clip]{./img/sample15.png}
      \caption{ログイン完了画面}
      \label{fig:図の名前}
    \end{center}
    \end{figure}

    \clearpage

\section{掲示板機能}

\section{チャット機能}


\chapter{結論}
\label {chp:tex_basic}
本研究では,社員の労務管理の支援のために顔写真から表情の成分情報を管理者に見える化する
システムの構築を目的とした. 実際にシステムの構成内容の検討から実装までを行う事で
FaceAPIを用いた表情分析を活用した労務管理システムを完成させることができた. \\
 開発にあたって,Reactを利用しMaterial Designを用いて,
慣れていない人でも使いやすいUIデザインにし,Material UIのコンポーネントを使用したことで,
可読性の高いソースコードにすることができた.
またFirebaseの機能である Firebase Authentication,Cloud Firestoreを
利用することで,バックエンドのサーバーの構築・保守が不要になった. \\
 今後の課題としては,iosやAndroidといったモバイルアプリでも使用できる様にすることが挙げられる.
今回作成したシステムがモバイルでも使用できれば,スマートフォンなどで取った写真をパソコンへ送る
作業をせず,そのまま提出できるため,使いやすさの向上に繋がる.


\newpage

%%%%%%%本文 end%%%%%%%%%%%%%%%%%%%%%%%%%%%%%%%%%%%%%%%%%%%%

%%%%%%%謝辞 start%%%%%%%%%%%%%%%%%%%%%%%%%%%%%%%%%%%%%%%%%%%

\chapter*{ \\謝辞}\addcontentsline{toc}{chapter}{謝辞}
本研究に関しまして,熱心かつ丁寧にご指導いただきました,千葉工業大学情報科
学研究科情報科学専攻中村直人教授に心から御礼申し上げます.また,中間審査並び
に最終審査にて副査を務めていただきました八島由幸教授, 眞部雄介教授,中川泰宏助教授に深く感謝いたします.そして,中村研究室での一年間のゼミナール課程及び,
学部生の皆様に心から御礼申し上げます.皆様のおかげで,これまでの研究生活を充
実かつ楽しく送ることが出来ました.そして,これまで支えてくれた母親をはじめと
する親族各位に改めて御礼申し上げます.ありがとうございました.

%%%%%%%謝辞 end%%%%%%%%%%%%%%%%%%%%%%%%%%%%%%%%%%%%%%%%%%%%

%%%%%%%参考文献 start%%%%%%%%%%%%%%%%%%%%%%%%%%%%%%%%%%%%%%%%%%%
\begin{thebibliography}{99}

\bibitem{パーソルホールディングス株式会社}
パーソルホールディングス株式会社,
\url{https://www.persol-group.co.jp/news/20211111_9085/} (2022 年 8 月 31 日)

\bibitem{厚生労働省}
厚生労働省,
\url{https://www.no-harassment.mhlw.go.jp/foundation/statistics/} (2022 年 8 月 31 日)

\bibitem{カオナビ}
カオナビ,
\url{https://www.kaonavi.jp/?utm_source=google&utm_medium=cpc&utm_campaign=google_cpc_A&gclid=CjwKCAiAhqCdBhB0EiwAH8M_Gv8ie7bWYXLeyRZ--DuOTmw6RfqsTd9O-9iZ8vv7j6GMSB6jDtxw1hoCRgQQAvD_BwE} (2022 年 8 月 31 日)

\bibitem{SmartHR}
SmartHR,
\url{https://smarthr.jp/?utm_source=google&utm_medium=cpc&utm_campaign=search-01-shimei&utm_term=c-smarthr-b&utm_content=69422661316-574814648627&gclid=CjwKCAiAhqCdBhB0EiwAH8M_Gm8xRn8FYsf-eqOMmc4K6Dl-eWRoRcuU1v438E5dezcEMsoCCuH1yxoC5wAQAvD_BwE} (2022 年 8 月 31 日)

\bibitem{freee}
freee,
\url{https://www.freee.co.jp/lp/brand/01/?utm_source=google&utm_medium=cpc&utm_content=3413057902_10458796026_105182885178_613511216136_aud-439291388342:kwd-296216324822&utm_campaign=01NQ_SCH_A10%28brand%29-%E6%8C%87%E5%90%8D&utm_term=e_freeee&referral=aw_brand&gclid=CjwKCAiAhqCdBhB0EiwAH8M_GruHVtShiDkJ4AptOAuUNcMCljeiqPzXLZFsZZ5_S9yGdqKq-8HbTxoCk-YQAvD_BwE} (2022 年 8 月 31 日)

\bibitem{GCP}
GCP,
\url{https://cloud.google.com/} (2022 年 8 月 31 日)

\bibitem{AWS}
AWS,
\url{https://aws.amazon.com/jp/} (2022 年 8 月 31 日)

\bibitem{Azure}
Azure,
\url{https://azure.microsoft.com/ja-jp/} (2022 年 8 月 31 日)

\bibitem{Firebase}
Firebase,
\url{https://firebase.google.com/?hl=ja} (2022 年 8 月 31 日)

\bibitem{Face API}
Face\hspace{3pt}API,
\url{https://azure.microsoft.com/ja-jp/products/cognitive-services/face/} (2022 年 8 月 31 日)

\bibitem{Amazon Rekognition}
Amazon\hspace{3pt}Rekognition,
\url{https://docs.aws.amazon.com/ja_jp/rekognition/latest/dg/what-is.html} (2022 年 8 月 31 日)

\bibitem{Cloud API}
Cloud Vision API,
\url{https://cloud.google.com/vision/?hl=ja} (2022 年 8 月 31 日)

\bibitem{face-api.js}
face-api.js,
\url{https://justadudewhohacks.github.io/face-api.js/docs/index.html} (2022 年 8 月 31 日)

\bibitem{React}
React,
\url{https://ja.reactjs.org/} (2022 年 8 月 31 日)

\bibitem{Material UI}
Material UI,
\url{https://mui.com/} (2022 年 8 月 31 日)

\bibitem{Firebase Authentication}
Firebase Authentication,
\url{https://firebase.google.com/docs/auth} (2022 年 8 月 31 日)

\bibitem{Cloud Firestore}
Cloud Firestore,
\url{https://firebase.google.com/docs/firestore} (2022 年 8 月 31 日)

\end{thebibliography}

\end{document}
