\chapter{機能設計}
\label{chp:chart}

\section{労務管理支援システム}
\label{sec:chart_figure}
今回作成するシステムは,会社で使うことを想定しており,労務管理者と社員で使える機能を分ける.

\subsection{労務管理システムとは}
労務管理システムとは,勤怠管理や年末調整,給与明細などの従業員情報の管理を単純化し,
人事業務を効率化するためのシステムである.
従来は,紙やエクセルで従業員の情報・書類を管理していたため,
従業員数に比例し負担が大きくなってしまっていた.
労務管理システムなら,スマホやwebで従業員に入力してもらった情報を自動で集約して
クラウド上で一元管理できる.以下に,現在多くの会社で使われている労務管理システムについて三つ
説明する.

\renewcommand{\labelenumi}{(\arabic{enumi})}

\begin{enumerate}
  \item カオナビ \\
   カオナビは,株式会社カオナビが運営する顔写真を用いることによって一元化された人材情報を見える化し簡単に共有できるクラウド型のプラットフォームである.
  社員の顔や名前,評価,スキルなどのあらゆる人材情報を一元管理して可視化することで,人材配置や評価運用の効率化,モチベーション分析など様々な人事課題を解決する.
  約2500社への導入した実績と蓄積されたノウハウを持っており、サポート体制も整っている.また他企業の人事同士が情報交換を行えるコミュニティがある.\\

  \item SmartHR\\
   SmartHRは株式会社SmartHRが運営するクラウド型の人事労務業務を効率化させるサービスである.
  Web上で従業員一人一人の人事評価や年末調整,雇用契約などの書類の作成から提出を簡単に行える.ペーパーレス化することによって,紙の書類を使う手間やコストを削減し,効率化できる.
  従業員から提出された書類やデータは,自動で更新され最新の人事データを一元化して管理できる.
  蓄積されたデータの可視化や分析が簡単になっており,効果的な人材マネジメントができる.
  外部の40以上のサービスとの連携可能であり,退勤管理,給与計算などの労務業務を効率化する. \\

  \item freee人事労務\\
   人事労務freeeとは,freee株式会社が提供している従業員の給与の計算から労務管理までを効率よく行えるクラウド型の人事労務管理システムである.
毎月の給与計算をはじめ給与明細発行や勤怠管理,年末調整,マイナンバー管理などの労務業務を自動化できる.自動化により,ミスなどの手間がなくなり作業時間を削減することができる.
他社サービスと連携することによって,従業員データを同期し給与計算、年末調整などを行うことが可能である.
\end{enumerate}

\section{クラウド}
\subsection{クラウドサービス}
クラウドコンピューティングサービスは,サーバやストレージ,ソフトウェアなどをネットワークを経由して利用するサービスである.
従来は,自社でサーバの設置から運用や保守を行いながら,アプリケーションの構築,運用保守をする必要があった.そのため初期コストやセキュリティなどを気にしなければいけなかった.
ネットワーク環境下でさえあればどこでも,クラウドサービスを利用することができ,これまでのサーバーやストレージなどへの初期投資が必要なくなり,
構築から保守までのさまざまな手間や時間を削減できる.
またコストの削減や高い拡張性,メンテナンスが不要などといったメリットが得られる.セキュリティ面においてもクラウド提供企業の高いセキュリティ技術によって守られている,
以下に代表的なクラウドについていくつか説明する.\\

\renewcommand{\labelenumi}{(\arabic{enumi})}

\begin{enumerate}
  \item Google Cloud(GCP) \\
   GCPは,Googleにより提供されているクラウドサービスである.
  GCPの特徴は,Google社内で使用されている技術がベースとなっており,セキュリティにおいても強固である,高速,そして安定した強固なインフラ環境・技術を利用できることである.
  強みとしてスプレッドシートやGmailなどのGoogleサービスとの連携がある.またデータ分析や機械学習に定評があり,画像分析やテキスト翻訳などのAPIサービスがある.\\

  \item Amazon Web Services(AWS) \\
   AWSは,Amazonにより提供されているクラウドサービスである.
  AWSは,セキュリティーも高く,クラウドの中でもトップシェアを誇っている.
  サービスの種類が200以上と豊富であるため,幅広い用途に活用することができAWSのみでシステムを構築することが可能である.
  日本語でのサポートが受けられる.
  また日本でも資格保有者も多数おり,日本語での情報やナレッジも比較的集めやすい.
  \\

  \item Microsoft Azure(Azure) \\
   Microsoft Azureは,Microsoftが提供しているクラウドサービスである.
  AWSに次ぐシェアを誇っている.
  WindowsやOfficeなどのMicrosoft系サービスやテクノロジーとの親和性が高い.
  オンプレミス環境との連携がしやすく,大企業や官公庁などでも使用されることの多い.\\

  \item  Firebase \\
   FirebaseはGoogleが提供しているモバイルおよびWebアプリケーションのバックエンドサービス
  である.クラウドサービスの形態ではBaaSに位置付けられる.Firebaseを使うことで,
  開発者はアプリケーションの開発に専念でき,バックエンドで動くサービスを作成する必要も
  管理する必要もない.サービスの早期リリースという要件が求められたときに,
  サーバレスアーキテクチャが注目され,BaaSというクラウドサービスの形態が登場した.
  サーバサイドの開発費を抑え,かつ工数もかからない.サービスの利用者が増えてもサーバの増築を
  意識しなくて良いなどの利点から,Webサービスほどサーバを必要としないモバイル向けのサービスの
  BaaSに注目され,様々なBaaSが登場した.
  Firebaseもその中の一つである.FirebaseはもともとはGoogleとは独立したサービスでしたが,
  2014年にGoogleに買収され,GCPの仲間入りをした.現在はGCPの様々なサービスと連携して使うことができるようになっている.
\end{enumerate}

\subsection{表情認識について}
表情認識とは,AIを活用して顔の表情を認識,分析し,その人の心理状態を推測する技術のことである.
最近では,マスクをしていても表情を認識できるようになって来ている.
顔認識や表情認識ができるサービスやライブラリーも増えている.以下にそのサービスやライブラリーをいくつか紹介する.

\renewcommand{\labelenumi}{(\arabic{enumi})}

\begin{enumerate}
  \item Face API \\
   MicrosoftさんのCognitive Serviceとして提供されている.
  Cognitiveは「認知」という意味で,見たり聞いたり理解したりといったことを,
  コンピュータが行ってくれるサービスである.Cognitive Serviceには「視覚」,「音声」,
  「知識」,「検索」,「言語」の5つのカテゴリがある.
  FaceAPIは「視覚」カテゴリに属するサービスで,画像データから顔を検出し,年齢,性別,感情などを
  取得できる.また,予め人物毎に顔画像を登録しておく事で,検出した顔が誰なのかを識別することが
  できる.\\

  \item Amazon Rekognition \\
   Amazon Rekognitionでは,イメージ分析とビデオ分析をアプリケーションに簡単に追加できる.
  Amazon Rekognition APIにイメージやビデオを指定するだけで,このサービスによってモモ,
  人物,テキスト,シーン,アクティビティを識別できる.不適切なコンテンツも検出できる.
  Amazon Rekognitionでは,高精度の顔分析,顔比較,顔検索機能も備えている.顔の検出,
  分析,比較は,ユーザー検証,カタログ作成,人数計数,公共安全など,多岐にわたって活用できる.
  Amazon Rekognitionは,Amazonのコンピュータ視覚科学者が日々何十億ものイメージを
  分析するために開発したものと同じ,実証済みで高度にスケーラブルな
  ディープラーニングテクノロジに基づいている.使用するのに機械学習の専門知識は必要ない.
  Amazon Rekognitionには,Amazon S3に保存されているイメージファイルやビデオファイルを
  即座に分析できる,シンプルで使いやすいAPIが含まれている.
  Amazon Rekognitionは絶えず新しいデータを学習させ,新たなラベルおよび顔比較機能を
  継続的にサービスに追加している.\\

  \item Cloud Vision API  \\
   REST APIやRPC APIを介して優れた事前トレーニング済み機械学習モデルを提供する.
  画像にラベルを割り当てることで,事前定義済みの数百万のカテゴリに画像を高速に分類できる.
  オブジェクトを検出し,印刷テキストや手書き文字を読み取り,有用なメタデータを画像カタログに
  作成する.\\

  \item face-api.js \\
   Face-api.jsは,ブラウザおよびNodeJSで顔を検出し認識するための,Tensorflow.jsを
  活用したJavaScript APIである.face-api.jsのソースコードは,
  GitHubに公開されており,インストールするだけで使うことができる.
  
\end{enumerate}


\section{提案するシステムの機能}
\label{sec:chart_admin}

\renewcommand{\labelenumi}{(\arabic{enumi})}

\begin{enumerate}
  \item 社員管理(管理者のみ) \\
   社員管理は社員の名前,メールアドレス,社員番号などを表で管理する.さらにその表から名前などを編集
  できる機能,社員が提出した写真を見る機能とその顔写真を表情分析したグラフを見る機能,過去5日分の
  笑顔の傾向をグラフで見る機能を実装する.  \\

  \item 掲示板 \\
   労務管理者が社員に何かお知らせする際に利用できる機能.
  掲示板で労務管理者が投稿をすると,全員がその投稿を見ることができる.
  労務管理者が社員に向けてお知らせする機能なので,労務管理者のみが投稿でき,
  社員は閲覧のみできる. \\

  \item チャット \\
   管理者と社員が一対一でコミュニケーションを取れる機能.
  (2)の掲示板とは異なり,個別で労務管理者からも社員もメッセージのやり取りができる. \\

  \item  顔写真提出(社員のみ) \\
   社員がその日に取った顔写真を提出する機能.この機能を使い顔写真を提出することで,
  (1)の社員管理で労務管理者が社員の顔写真を見ることができる.もし顔を認識できない写真
  だった場合,もう一度,顔写真を提出させる機能も実装する.
\end{enumerate}

