\chapter{機能設計}
\label{chp:chart}

\section{FaceAPIを用いた労務管理支援システム}
\label{sec:chart_figure}
今回作成するシステムは,会社で使うことを想定しており,労務管理者と社員で使える機能を分ける.

\subsection{労務管理システムとは}
労務管理システムとは,勤怠管理や年末調整,給与明細などの従業員情報の管理を単純化し,
人事業務を効率化するためのシステムである.
従来は,紙やエクセルで従業員の情報・書類を管理していたため,
従業員数に比例し負担が大きくなってしまっていた.
労務管理システムなら,スマホやwebで従業員に入力してもらった情報を自動で集約して
クラウド上で一元管理できる.以下に,現在多くの会社で使われている労務管理システムについて三つ
説明する.

\renewcommand{\labelenumi}{(\arabic{enumi})}

\begin{enumerate}
  \item カオナビ \\
   カオナビは,顔写真が並ぶシンプルな画面から一元化された人材情報をクラウド上で
  簡単に共有できるプラットフォームである.社員の顔や名前,経験,評価,スキル,才能などの
  人材情報を一元管理して可視化することで,最適な人材配置や抜擢といった人材マネジメント業務を
  サポート.あらゆる人材マネジメントの課題を解決し,企業の働き方改革を実現する
  HRテクノロジーサービスとして,2012年のローンチ以降,業種・業態を問わず1,100社以上の経営者や
  現場のマネジメント層に選ばれている. \\

  \item SmartHR\\
   SmartHRは従業員データの収集から効率的に進めることができる.
  SmartHRは人事マスターデータの入力を原則従業員側でスマホなどのモバイル端末で行い,
  人事部は入力されたデータを従業員名簿として一元的に管理できる.
  入社時点から一貫したペーパレス管理で業務効率が大幅にアップする.
  蓄積されたデータは,人事労務各種手続きに利用され,ワークフローによるペーパレス申請の
  社内手続きだけでなく,入力情報を社保手続きに利用可能であり,
  外部の豊富なサービスとも連携が可能である. \\

  \item freee人事労務\\
   人事労務freeeとは,従業員の給与の計算から労務管理までをスムーズに行えるクラウド型の
  人事労務管理システムであり,freee株式会社が提供している.
人事労務freeeを導入することで,毎月の給与計算をはじめ給与明細発行や勤怠管理,年末調整,
マイナンバー管理などの労務業務の一元化を実現することができる.
雇用保険料や社会保険料,所得税といった計算なども自動で行ってくれるため,
労務業務で起こりやすい人的ミスの軽減に加え,業務フローの効率化につなげることができる.
\end{enumerate}

\subsection{クラウドサービス}
クラウドサービスは,従来は利用者が手元のコンピュータで利用していたデータやソフトウェアを,
ネットワーク経由で,サービスとして利用者に提供するものである.
利用者側が最低限の環境(パーソナルコンピュータや携帯情報端末などのクライアント,
その上で動くWebブラウザ,インターネット接続環境など)を用意することで,どの端末からでも
さまざまなサービスを利用することができる.
これまで,利用者はコンピュータのハードウェア,ソフトウェア,データなどを
自身で保有・管理し利用していたが,クラウドサービスを利用することで,
これまで機材の購入やシステムの構築,管理などにかかるとされていたさまざまな手間や時間の削減を
はじめとして、業務の効率化やコストダウンを図れる.
以下に代表的なクラウドについていくつか説明する.

\renewcommand{\labelenumi}{(\arabic{enumi})}

\begin{enumerate}
  \item Google Cloud(GCP) \\
  GCPは,Googleにより提供されるクラウドサービスである.
  GCPの最大の特徴は,Google社内で使用されている,安定した強固なインフラ環境・技術を
  利用できることである.加えて,機械学習やAI開発,ビッグデータの高速分析にも定評がある.
  世界中にGoogle専用のネットワークが敷設されていて,グローバル展開に強いサービスである.  \\

  \item Amazon Web Services(AWS) \\
  AWSは,Amazonにより提供されるクラウドサービスである.
  AWSは,サービスの種類が豊富であるため,幅広い用途に活用することができる.
  AWSを扱うことができる技術者の数も多く,日本語での情報やナレッジも比較的集めやすい.\\

  \item Microsoft Azure(Azure) \\
  Microsoft Azureは,Microsoftにより提供されるクラウドサービスである. 
  OfficeをはじめとするMicrosoft系サービスやテクノロジーとの親和性が高い.
  オンプレミス環境との連携がしやすく,大企業や官公庁などでも使用されることの多い.\\

  \item  Firebase \\
\end{enumerate}

\subsection{表情認識について}

\section{システムの機能}
\label{sec:chart_admin}

\renewcommand{\labelenumi}{(\arabic{enumi})}

\begin{enumerate}
  \item 社員管理(管理者のみ) \\
  社員管理は社員の名前,メールアドレス,社員番号などを表で管理する.さらにその表から名前などを編集
  できる機能を実装する.  \\

  \item 掲示板 \\
  労務管理者が社員に何かお知らせする際に利用できる.管理者しか発言できないようになっている. \\

  \item チャット \\
  管理者と社員が一対一でコミュニケーションを取れるツール.相互からメッセージのやり取りができる. \\

  \item  顔写真提出(社員のみ) \\
  社員がその日に取った顔写真を提出する.
\end{enumerate}

